\documentclass[a4paper]{article}
%\documentclass[a4paper,10pt]{article}
\usepackage{geometry}
 \geometry{
 a4paper,
 total={210mm,297mm},
 left=20mm,
 right=20mm,
 top=20mm,
 bottom=20mm,
 }
\usepackage[utf8]{inputenc}
\usepackage[english,czech]{babel}
\usepackage{makeidx}
\usepackage{url}
\usepackage{tikz}
\usepackage{float}
\usepackage{pdfpages}
\usepackage{amsfonts}
\usepackage{mdwlist}
\usepackage{xcolor}
\usepackage{listings}
\usepackage[utf8]{inputenc}
\usepackage[T1]{fontenc}
\usepackage{listingsutf8}
\usepackage{cite}
\usepackage{mdframed}
\usepackage[affil-it]{authblk}

\begin{document}
\title{Basic Pollard Rho Algorithm Implementation On \texttt{CUDA} Device}
\author{Martin Beránek}
\date{\today}
\affil{Faculty of Information Technology -- Czech Technical University in Prague}
\maketitle

%\tableofcontents
%\listoffigures
%\listoftables

\section{Introduction}

Factorisation problem of a huge number resolved into massive parallel solutions. Large number of algorithms are currently state-of-art and are continuously developed into better forms. This short article is focused on implementation of Pollard-Rho algorithm on \texttt{CUDA} device. In first part there is a definition of algorithm. Next the article focuses on options of parallelism on CUDA device. Results are measured in multiple instances and compared in graphs. 

\section{Definition of the algorithm}


The $\rho$ algorithm (named after the shape of curves symbolising two functions trying to reach themselves in projective space) is based on finding cycle. In t random numbers of $x_1, x_2, \dots, x_t$ in range $[1, n]$ will contain repetition with probability of $P > 0.5$ if $t > 1.777n^{\frac{1}{2}}$


\section{\texttt{CUDA} Solution}


\subsection{First solution with explicit barrier}


\subsection{Second solution with independent runners}


\section{Conclusion}

\bibliographystyle{iso690}
\bibliography{mybibliographyfile}


\appendix

\end{document}
